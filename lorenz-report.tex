\documentclass{report}
\begin{document}
\chapter{Introduction to Chaos in the Lorenz System}
\section{Preamble}
In higher dimensions, nonlinear systems of differential equations have the
potential to act in ways not possible in one or two-dimensional systems. In 
two-dimensional systems, for instance, any trajectory that can be contained within a closed, bounded subset of the plane will either approach a fixed point
(if it exists), or, if the fixed point is excluded from R, the trajectory is
either a closed orbit or approaches one asymptotically. In effect, this proves
that if we can 'trap' a trajectory within a bounded region, then we expect its 
long-term behavior to be periodic or fixed.
	
As we move to three dimensions this result no longer holds. A solution to a
dynamical system can be shown to stay within a closed, bounded region yet
never settle into any predictable long-term behavior. Furthermore, solutions 
that begin with similar initial conditions may actually fly apart in the 
system, so that it becomes difficult to extrapolate the motion of a trajectory 
from the starting point. These traits fall under the definition of chaos in a
dynamic system. The goal of our report is to explore the basics of dynamical 
chaos through an analysis of the Lorenz equations, relying on computer
simulation when deriving the behavior of the system by hand proves intractable.

\section{Stability Analysis}

We first state the Lorenz equations:
\[ x = y\]

\end{document}
