\documentclass{report}
\begin{document}
	In higher dimensions, nonlinear systems of differential equations have the potential to
	act in ways not possible in one or two-dimensional systems. In two-dimensional systems,
	for instance, any trajectory that can be contained within a closed, bounded subset of the
	plane will either approach a fixed point (if it exists), or, if the fixed point is
	excluded from R, the trajectory is either a closed orbit or approaches one asymptotically.
	In effect, this proves that if we can 'trap' a trajectory within a bounded region, then we
	expect its long-term behavior to be periodic or fixed.
	
	As we move to three dimensions this result no longer holds. A solution to a dynamical
	system can be shown to stay within a closed, bounded region yet never settle into any
	predictable long-term behavior.
\end{document}