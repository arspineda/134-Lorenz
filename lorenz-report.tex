\documentclass{report}
\usepackage{amsmath}
\begin{document}
\chapter{Introduction to Chaos in the Lorenz System}
\section{Preamble}
In higher dimensions, nonlinear systems of differential equations have the
potential to act in ways not possible in one or two-dimensional systems. In 
two-dimensional systems, for instance, any trajectory that can be contained within a closed, bounded subset of the plane will either approach a fixed point
(if it exists), or, if the fixed point is excluded from R, the trajectory is
either a closed orbit or approaches one asymptotically. In effect, this proves
that if we can 'trap' a trajectory within a bounded region, then we expect its 
long-term behavior to be periodic or fixed.
	
As we move to three dimensions this result no longer holds. A solution to a
dynamical system can be shown to stay within a closed, bounded region yet
never settle into any predictable long-term behavior. Furthermore, solutions 
that begin with similar initial conditions may actually fly apart in the 
system, so that it becomes difficult to extrapolate the motion of a trajectory 
from the starting point. These traits fall under the definition of chaos in a
dynamic system. The goal of our report is to explore the basics of dynamical 
chaos through an analysis of the Lorenz equations, relying on computer
simulation when deriving the behavior of the system by hand proves intractable.

\section{Stability Analysis}

We first state the Lorenz equations:
\begin{align*}
   \dot{x} = \sigma y - \sigma x \\
   \dot{y} = rx - y - xz \\
   \dot{z} = xy - bz
\end{align*}
where \( \sigma , b, and r \) are parameters greater than zero. Our first goal is to
locate the fixed points of the system and identify their behavior. Setting
\( \dot{x}, \dot{y}, and \dot{z}\) to zero, we have
\begin{align*}
   x = y \\
   (r-z)x - y = 0 \\
   xy = bz
\end{align*}
Replacing y with x as given in the first equation:
\begin{align*}
  x = y \\
  (r-z-1)x = 0 \longrightarrow x = 0 or z = r-1 \\
  x^2 = bz
\end{align*}

We see that three fixed points are possible: \( \boldsymbol{x\mbox{*}} = (0,0,0),(\sqrt{b(r-1)},\sqrt{b(r-1)},r-1)\), and \((-\sqrt{b(r-1)},-\sqrt{b(r-1)},r-1)\).
The fixed point at the origin exists for all choices of r, while the symmetric 
pair of fixed points will only exist if \(r > 1\). In fact, a supercritical 
pitchfork bifurcation occurs at the origin when \( r = 1 \).

The Jacobian matrix for the Lorenz system i
\[
\begin{bmatrix}
   -\sigma & \sigma & 0 \\
   r - z & -1 & -x \\
   y & x & -b
\end{bmatrix}
\]

For the fixed point at the origin, the linearization of the system becomes
\[
\begin{bmatrix}
   -\sigma & \sigma & 0 \\
   r & -1 & 0 \\
   0 & 0 & -b

\end{bmatrix}
\]
The z component decays exponentially, while the behavior of x and y depends on 
r: for \( r < 1\), the determinant of the matrix containing only x and y is
\(\sigma - r\sigma > 0\), and the trace is negative. We find the discriminant 
\(tr^2 - 4det = \sigma^2 +2\sigma + 1 -4(\sigma -r\sigma) = (\sigma - 1)^2 +
  4r\sigma\). Since r and \(\sigma > 0 \), we have \(tr^2 -4det > 0\), which 
means that the fixed point at the origin is a stable node. When \(r > 1\), on 
the other hand, the origin changes stability and becomes a saddle node, and the
fixed points \(C^+\) and \(C^-\) appear. If this is a supercritical pitchfork
bifurcation we expect these to be stable.

The linearization of \(C^+\) and \(C^-\) is as follows:
\[
\begin{bmatrix}
   -\sigma & \sigma & 0 \\
   1 & -1 & -d \\
   d & d & -b
\end{bmatrix}
\]
Where \(d = \pm \sqrt{b(r-1)}\). When we derive the characteristic polynomial 
for this linearization we find that they are identical, so we only show the
process for positive \(d\). Now we see that the behavior of z is no longer
decoupled from x or y, so we cannot simplify the analysis to the 2-dimensional
case. Instead, we find the eigenvalues directly:
\begin{align*}
  det(A-\lambda I) = (-\sigma-\lambda)((-1-\lambda)(-b-\lambda)+d^2 )-
  \sigma(-b-\lambda+d^2) \\
  \quad = (\lambda^2 +(\sigma+1)\lambda+\sigma)(-b-\lambda)+
d^2(-\sigma-\lambda)+b\sigma+\sigma\lambda-\sigma d^2 \\
  \quad = (-1)(\lambda^3 +(\sigma + b + 1)\lambda^2 +(b\sigma+\sigma+b)\lambda
+b\sigma)+b\sigma+(\sigma-d^2)\lambda-2\sigma d^2
\end{align*}
We let \(det(A-\lambda I) = 0\), which allows us to simplify the polynomial
into its most useful form:
\begin{align*}
  det(A-\lambda I) = \lambda^3 + (\sigma+b+1)\lambda^2 + b(\sigma+r)\lambda
+ 2\sigma b(r-1) = 0
\end{align*}
Finding the eigenvalues for this linearization for certain parameter values
is computationally laborious, but possible. When we set \(\sigma = 10\),
\(b=8/3\), and vary our choice of \(r\), we find that for some r we get
complex eigenvalues with negative real part, implying that \(C^+\) and \(C^-\)
are stable spirals at some point. To show this analytically, we will actually
find the value of \(r\) at which a pair of purely imaginary solutions appear.
We call this value \(r_H\), and by deriving this we learn that \(C^+\) and
\(C^-\) undergo Hopf bifurcations. 

Suppose we do have a pair of purely imaginary eigenvalues \(\lambda = \pm
i\omega\), where \(\omega\) is real, and that \( \sigma > b+1 \). Then the
characteristic polynomial is re-expressed as a complex number (it suffices
just to use the positive imaginary eigenvalue):
\begin{align*}
   -i\omega^3 - (\sigma + b + 1)\omega^2 + ib(\sigma + r)\omega
+ 2\sigma b(r-1) = 0 \\
   \rightarrow i(-\omega^3 +\omega b(\sigma + r) + 2\sigma b(r-1) -
(\sigma+b+1)\omega^2 = 0
\end{align*}
Since \(\omega\) and the parameters are all real, the above implies both
the real and imaginary parts of the equation must equal 0 as well. From this
we find a pair of equations:
\begin{align}
   -\omega^3 +\omega b(\sigma+r) = 0 \\
   -(\sigma +b+1)\omega^2 + 2\sigma b(r-1) = 0
\end{align}

\end{document}
